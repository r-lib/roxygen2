\documentclass{mjaeartcl}

\usepackage[x11names]{xcolor}

\usepackage{listings}

\lstset{
	language=S,
	showstringspaces=false,
	basicstyle=\ttfamily\small,
	commentstyle=\color{Blue4},
	belowskip=0pt}


\begin{document}

\wparea{Google Summer of Code 2008 Idea}
\title{Roxygen}

\makewptitle

\texttt{Roxygen} is an extension of the \texttt{R} language and
provides a documentation and development assistance system. The
documentation system is in the style of
\href{http://www.stack.nl/~dimitri/doxygen/}{Doxygen} or
\href{http://java.sun.com/j2se/javadoc/}{Javadoc}, and the development
assitance system is (somehow) like a pre-processor with its
directives.

A Roxygen block is marked as
\begin{lstlisting}
#'
\end{lstlisting}
and consists of one ore more lines. The blocks are processed by
Roclets (in the style of Doclets), which are small
\texttt{R}-programs, specifying the content and format of the output.


\section*{An outline}

A roxygen block must precede a class, method, generic function or
function declaration. It is made up of two parts: a description
followed by block tags.

An exemplar \texttt{S4} class with a generic function and the
corresponding method:

\begin{lstlisting}
#' This class represents a person.
#' 
#' @slot fullname The full name of the person
#' @slot birthyear The year of birth
#' @prototype Prototype person is named John Doe
#'     and born in the year 1971}
setClass('Person',
	representation = representation(
		fullname='character',
		birthyear='numeric'),
	prototype = prototype(
		fullname='John Doe',
		birthyear='1971'))
\end{lstlisting}

\begin{lstlisting}
#' The naming of an object.
#'
#' @param object A object which gets a name
setGeneric('name', function(object, ...){[...]})

#' Name a person, the baptism.
#'
#' @param object A Person object
#' @param ... Not used
#' @export
setMethod('name', signature('Person'),
function(object, ...) {
	...
})
\end{lstlisting}

The class description says the purpose of the class, and the tags
describe slots and the prototype object. The generic and method
arguments are described by \texttt{param} tags. The standard
Documentation-Roclet generates \texttt{Rd} files out of these
information.

The \texttt{export} tag in the method Roxygen block is a development
assistance tag; it defines that this method is exported. Another tag
is the \texttt{import} tag, which defines the package, where a
generic function or class is imported from (\texttt{@import
Package}). A Namespace-Roclet handles these information and creates
the package namespace file.

A Collate-Roclet handles multiply source files. At the beginning of
every file, the \texttt{@include} tag describes which other source
files must be visible to this one. The order of the source files is
calculated based on that tags, and the collate field of the package
description file is set.


\section*{Project attributes}

Sophisticated project; student needs to know \texttt{R} very well,
especially the structural elements like functions, S3/4 classes,
generics, ..., and the student needs some basic knowledge about
writing a parser. At best the student has worked with Doxygen, Javadoc
or another documentation system.

The project goals and milestones, respectivly, are
\begin{enumerate}
	\item Define a list of tags which are meaningfull in conjunction
	with \texttt{R}.

	\item Develop code to parse \texttt{R} source files: \texttt{R}
	code and Roxygen block comments.

	\item Define the behavior of a Roclet; write the documentation
	Roclet which produces \texttt{Rd}-files.
\end{enumerate}



\end{document}
